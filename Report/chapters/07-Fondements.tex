\chapter{Fondements théoriques et contexte financier}
\label{chap:prerequis}

Ce chapitre présente les concepts financiers et mathématiques fondamentaux nécessaires à la compréhension de ce projet. Ces connaissances constituent le socle théorique des modèles de valorisation des options ainsi que des méthodes numériques utilisées.

\section{Marchés financiers et produits dérivés}
\subsection{Structure et fonctionnement des marchés}

\subsubsection{Marchés organisés (Bourses)}
Un marché organisé, ou bourse, est une plateforme centralisée où acheteurs et vendeurs effectuent des transactions sur des instruments financiers normalisés. Les ordres sont transmis via des intermédiaires à des membres officiels de la bourse. Une chambre de compensation intervient comme contrepartie unique, garantissant ainsi l'exécution des transactions et éliminant le risque de contrepartie.

Les principales caractéristiques des marchés organisés sont :

\begin{itemize}
	\item Standardisation des contrats et procédures,
	\item Centralisation des ordres d'achat et de vente,
	\item Accès réservé aux membres agréés,
	\item Présence d'une chambre de compensation.
\end{itemize}

Aujourd'hui, les échanges s'effectuent principalement de manière électronique, remplaçant progressivement les transactions physiques sur les parquets.

\subsubsection{Marchés de gré à gré (OTC)}
Les marchés de gré à gré (Over-The-Counter, OTC) sont décentralisés : les transactions s'effectuent directement entre deux contreparties, sans chambre de compensation. Cette configuration implique un risque de contrepartie plus élevé et une moindre transparence.

Les marchés OTC sont animés par des \textit{market makers} qui fournissent en continu des cotations (bid et ask). La fragmentation de ces marchés peut entraîner des écarts de prix, exploitables par des stratégies d'arbitrage.

\subsection{Le marché des options}

\subsubsection{Historique}
Le premier marché moderne d'options a été créé en 1973 au \textit{Chicago Board of Options Exchange} (CBOE), parallèlement à la publication de l'article fondateur de Black et Scholes : \textit{"The Pricing of Options and Corporate Liabilities"}. En Europe, le \textit{London Traded Option Exchange} a ouvert en 1978, suivi en 1987 par le \textit{MONEP} à Paris.

\subsubsection{Fonctionnement du marché des options}
Les marchés organisés d'options assurent :

\begin{itemize}
	\item Une liquidité continue,
	\item Une transparence des prix,
	\item Une gestion du risque de contrepartie via la chambre de compensation.
\end{itemize}

Les acteurs principaux sont :

\begin{itemize}
	\item \textbf{Brokers} : exécutent les ordres pour le compte des investisseurs,
	\item \textbf{Market makers} : assurent la liquidité par des cotations permanentes,
	\item \textbf{Chambres de compensation} : garantissent les transactions en demandant des dépôts de garantie et des appels de marge.
\end{itemize}

\subsubsection{Types d'options}
On distingue :

\begin{itemize}
	\item \textbf{Options vanilles} :
	\begin{itemize}
		\item \textit{Européennes} : exerçables uniquement à maturité,
		\item \textit{Américaines} : exerçables à tout moment jusqu'à l'échéance.
	\end{itemize}
	\item \textbf{Options exotiques} : à barrière, lookback, asiatiques, etc.
\end{itemize}

Le payoff pour une option européenne est donné par :

\begin{align*}
	\text{Call :} & \quad H_T = (S_T - K)^+ \\
	\text{Put :} & \quad H_T = (K - S_T)^+
\end{align*}

\subsubsection{Terminologie essentielle}
Quelques termes clés :

\begin{itemize}
	\item \textbf{Prime} : coût initial pour acquérir l'option,
	\item \textbf{Sous-jacent} ($S$) : actif lié à l'option,
	\item \textbf{Prix d'exercice} ($K$) : prix convenu pour l'achat ou la vente,
	\item \textbf{Échéance} ($T$) : date limite d'exercice,
	\item \textbf{Valeur intrinsèque} : gain immédiat si l'option est exercée,
	\item \textbf{In The Money (ITM)} : option avec valeur intrinsèque positive,
	\item \textbf{Out of The Money (OTM)} : option avec valeur intrinsèque nulle,
	\item \textbf{At The Money (ATM)} : prix du sous-jacent proche du prix d'exercice.
\end{itemize}

\section{Risque et volatilité}

\subsection{Définition et types de risques de marché}
En finance, le risque est la possibilité de variations imprévues dans les résultats financiers.

Les principaux risques de marché sont :

\begin{itemize}
	\item Risque de taux d'intérêt,
	\item Risque de change,
	\item Risque actions,
	\item Risque sur matières premières,
	\item Risque de crédit,
	\item Risque de volatilité.
\end{itemize}

\subsection{La volatilité comme mesure du risque}

\subsubsection{Volatilité historique}
Basée sur les fluctuations passées du sous-jacent, la volatilité historique est utile pour l'analyse rétrospective, mais limitée pour prévoir les comportements futurs.

\subsubsection{Volatilité implicite}
La volatilité implicite est extraite du prix des options observées sur le marché en utilisant, par exemple, le modèle de Black-Scholes. Elle traduit les anticipations du marché quant aux variations futures de l'actif sous-jacent.

\section{Concepts avancés pour l'analyse des options}

\subsection{Univers risque neutre}

\subsubsection{Martingale}
Un processus stochastique $(M_t)_{t \geq 0}$ est une martingale par rapport à l'information $\mathcal{F}_t$ si :

\[
\mathbb{E}[M_t \mid \mathcal{F}_s] = M_s, \quad \forall s < t
\]

Cela signifie que, compte tenu de l'information à l'instant $s$, la meilleure estimation de $M_t$ est $M_s$.

\subsubsection{Univers risque neutre}
Dans un univers risque neutre, les prix actualisés des actifs sont des martingales sous une probabilité risque-neutre. Cela facilite la valorisation des produits dérivés sans nécessiter de primes de risque.

\subsection{Les "Grecques" : indicateurs de sensibilité}
Les \og grecques \fg{} mesurent la sensibilité du prix d'une option aux facteurs de marché :

\begin{itemize}
	\item Delta : variation par rapport au sous-jacent,
	\item Gamma : variation du Delta,
	\item Vega : variation par rapport à la volatilité,
	\item Theta : variation par rapport au temps,
	\item Rho : variation par rapport aux taux d'intérêt.
\end{itemize}

Ces indicateurs permettent une gestion efficace des risques.

\subsection{Les données de marché}
Yahoo Finance fournit des données de marché essentielles telles que les prix des actifs, les nappes de volatilité implicite et les courbes de taux d'intérêt utilisées pour la calibration et la valorisation des options.

