\chapter*{Introduction générale}
\addcontentsline{toc}{chapter}{Introduction générale}
\markboth{Introduction générale}{Introduction générale}
\label{sec:Introduction}

\section*{Contexte et importance du pricing d'options}

Les produits dérivés représentent aujourd'hui un marché colossal de plusieurs centaines de billions de dollars, jouant un rôle crucial dans la finance mondiale. Au cœur de ces instruments complexes, les options financières sont des outils stratégiques essentiels pour la gestion des risques, l'optimisation des portefeuilles et l'élaboration de stratégies spéculatives. Une option donne à son détenteur le droit (mais pas l'obligation) d'acheter (call) ou de vendre (put) un actif à un prix fixé d'avance (strike), soit à une date précise (option européenne), soit durant une période donnée (option américaine).

Le modèle de Black-Scholes, bien qu'élégant mathématiquement, se heurte à des limites importantes dans la pratique, notamment l'hypothèse d'une volatilité constante qui contredit clairement les observations empiriques des marchés. Face à ces anomalies, le modèle de Heston (1993) s'est imposé comme une alternative incontournable. En traitant la volatilité comme un processus stochastique corrélé au prix du sous-jacent, ce modèle offre un cadre plus souple capable de reproduire la plupart des caractéristiques observées sur les marchés, notamment les phénomènes de smile et de skew de volatilité.

Dans le contexte actuel des marchés financiers dynamiques, disposer d'outils efficaces et précis pour valoriser les options et visualiser l'impact des différents paramètres devient essentiel. Cependant, la complexité des modèles avancés comme celui de Heston et les différentes méthodes numériques disponibles peuvent rendre leur utilisation difficile pour les praticiens. Une interface interactive centralisant ces fonctionnalités tout en permettant une comparaison directe entre les approches représente donc une contribution significative tant pour la formation que pour la pratique professionnelle.

\section*{Problématique et objectifs du projet}

Ce projet se concentre sur le développement d'une interface interactive pour le pricing des options européennes, en mettant l'accent sur la comparaison entre différents modèles et méthodes numériques. Notre problématique s'articule autour de plusieurs questions fondamentales :

\begin{itemize}
	\item Comment implémenter efficacement et comparer les modèles de Black-Scholes et de Heston pour la valorisation d'options européennes ?
	
	\item Quelles sont les différences de performance et de précision entre les méthodes analytiques et les approches Monte Carlo basées sur le schéma d'Euler-Maruyama ?
	
	\item Comment optimiser le processus de calibration du modèle de Heston pour garantir une reproduction fidèle des prix observés sur le marché ?
	
	\item De quelle manière concevoir une interface utilisateur intuitive permettant d'explorer l'impact des différents paramètres sur la valorisation des options ?
\end{itemize}

Les objectifs de ce projet sont multiples :

\begin{enumerate}
	\item Développer un outil interactif complet pour le pricing d'options européennes, intégrant les modèles de Black-Scholes et de Heston.
	
	\item Implémenter et comparer différentes méthodes de pricing : solution analytique fermée (Black-Scholes), approche semi-analytique par fonctions caractéristiques (Heston) et simulations Monte Carlo avec schéma d'Euler-Maruyama (pour les deux modèles).
	
	\item Intégrer un module de calibration pour le modèle de Heston permettant d'ajuster les paramètres du modèle aux données de marché.
	
	\item Concevoir une interface graphique intuitive permettant aux utilisateurs de visualiser les résultats et d'explorer l'impact des paramètres sur les prix et les grecques des options.
	
	\item Offrir la possibilité d'utiliser des données de marché réelles pour enrichir l'expérience utilisateur et faciliter l'application pratique de l'outil.
\end{enumerate}

\section*{Méthodologie générale}

Notre approche méthodologique s'articule en quatre phases principales :

Premièrement, nous établissons les fondements théoriques nécessaires à la compréhension des modèles de pricing d'options, en expliquant les concepts fondamentaux du calcul stochastique et de la valorisation risque-neutre.

Deuxièmement, nous implémentons les modèles de Black-Scholes et de Heston, en détaillant leurs formulations mathématiques et leurs propriétés. Pour chaque modèle, nous développons à la fois des méthodes analytiques ou semi-analytiques et des approches par simulation Monte Carlo.

Troisièmement, nous nous concentrons sur la calibration du modèle de Heston, en implémentant des méthodes d'optimisation permettant d'ajuster les cinq paramètres du modèle (vitesse de retour à la moyenne, niveau moyen de variance à long terme, volatilité de la volatilité, corrélation, et variance initiale) aux prix d'options observés sur le marché.

Enfin, nous concevons l'architecture de l'interface utilisateur et implémentons les fonctionnalités de visualisation et d'interaction, en veillant à offrir une expérience intuitive qui facilite la comparaison entre les modèles et les méthodes.

Cette méthodologie structurée nous permet de développer un outil complet qui conjugue rigueur théorique et accessibilité pratique, tout en offrant une flexibilité pour différents cas d'utilisation, de l'enseignement à l'application professionnelle.

\section*{Structure du mémoire}

Ce mémoire s'organise en cinq chapitres :

Le présent chapitre d'introduction présente le contexte, les objectifs et la méthodologie générale du projet.

Le \textbf{Chapitre 1} expose les fondements théoriques nécessaires à la compréhension des modèles de pricing d'options. Il aborde les concepts essentiels comme le mouvement brownien, le calcul stochastique d'Itô, et les principes de valorisation risque-neutre, constituant la base mathématique des modèles plus avancés.

Le \textbf{Chapitre 2} détaille les modèles de pricing implémentés dans notre interface : le modèle de Black-Scholes et le modèle de Heston. Pour chaque modèle, nous présentons les hypothèses, les équations différentielles stochastiques sous-jacentes, et les solutions analytiques ou semi-analytiques pour la valorisation des options européennes.

Le \textbf{Chapitre 3} se concentre sur la calibration du modèle de Heston, une étape cruciale pour son utilisation pratique. Nous y présentons les méthodes d'optimisation implémentées, les défis spécifiques liés à la calibration de ce modèle à cinq paramètres, et les stratégies pour garantir la stabilité et la pertinence des résultats.

Le \textbf{Chapitre 4} décrit l'implémentation technique de l'interface interactive, incluant l'architecture logicielle, les choix technologiques, les algorithmes de pricing et de calibration, ainsi que les fonctionnalités de visualisation et d'interaction avec l'utilisateur.

Enfin, la \textbf{Conclusion} synthétise les contributions du projet, discute des avantages et des limitations de l'approche adoptée, et présente des perspectives d'amélioration et d'extension de l'interface développée.