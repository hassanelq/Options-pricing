\mychapter{0}{Résumé}
Ce projet présente une interface interactive dédiée au pricing des options européennes, offrant une approche comparative entre différents modèles et méthodes numériques. Nous implémentons le modèle de Black-Scholes sous sa forme analytique fermée ainsi qu'avec une approche Monte Carlo utilisant le schéma d'Euler-Maruyama. En parallèle, nous développons une implémentation complète du modèle de Heston, incluant à la fois un module de calibration et deux méthodes de pricing: la méthode semi-analytique par fonctions caractéristiques et l'approche Monte Carlo avec schéma d'Euler-Maruyama. L'interface permet aux utilisateurs de comparer directement les performances et précisions des différentes approches, d'explorer l'impact des paramètres sur la valorisation des options, et de visualiser les résultats à travers des graphiques dynamiques. Cette plateforme constitue un outil pédagogique et professionnel permettant d'approfondir la compréhension des modèles à volatilité constante et stochastique, tout en offrant des capacités de pricing en temps réel avec des données de marché réelles.
\vspace{1cm}

\noindent\rule[2pt]{\textwidth}{0.5pt}
{\textbf{Mots clés :}}
Options européennes, modèle de Black-Scholes, modèle de Heston, pricing d'options, volatilité stochastique, méthode semi-analytique, simulation Monte Carlo, schéma d'Euler-Maruyama, calibration, interface interactive
\\
\noindent\rule[2pt]{\textwidth}{0.5pt}
\clearpage

\mychapter{0}{Abstract}
This project presents an interactive interface dedicated to the pricing of European options, offering a comparative approach between different models and numerical methods. We implement the Black-Scholes model in its closed-form analytical solution as well as with a Monte Carlo approach using the Euler-Maruyama scheme. In parallel, we develop a complete implementation of the Heston model, including both a calibration module and two pricing methods: the semi-analytical method using characteristic functions and the Monte Carlo approach with Euler-Maruyama scheme. The interface allows users to directly compare the performance and accuracy of different approaches, explore the impact of parameters on option valuation, and visualize results through dynamic graphs. This platform serves as both an educational and professional tool for deepening the understanding of constant and stochastic volatility models, while offering real-time pricing capabilities with actual market data.
\vspace{1cm}

\noindent\rule[2pt]{\textwidth}{0.5pt}
{\textbf{Keywords :}}
European options, Black-Scholes model, Heston model, option pricing, stochastic volatility, semi-analytical method, Monte Carlo simulation, Euler-Maruyama scheme, calibration, interactive interface
\\
\noindent\rule[2pt]{\textwidth}{0.5pt}
\clearpage